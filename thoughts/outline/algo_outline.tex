\documentclass[12pt]{article}
\usepackage{fancyhdr}
\usepackage{xfrac}
\usepackage{extramarks}
\usepackage{amsmath}
\usepackage{amsthm}
\usepackage{amsfonts}
\usepackage{graphicx}
\usepackage{color}
\usepackage{pdfpages}
\usepackage{natbib}
\usepackage{caption}
\usepackage{subcaption}
\usepackage{hyperref}
\hypersetup{
    colorlinks=true,
    linkcolor=black,
    filecolor=black,      
    urlcolor=black,
    citecolor=black,
}
\usepackage{booktabs}
\usepackage{geometry}

\geometry{margin=1in}


%
% Basic Document Settings
%

%\topmargin=-1.5in
%\bottommargin=1.5in
%\evensidemargin=1.5in
%\oddsidemargin=1.5in
%\textwidth=6.5in
%\textheight=9.0in
%\headsep=0.25in

\linespread{1.5}

\pagestyle{plain}

% Paragraph indent length
%\setlength\parindent{0pt}

% Custom definitions
% For derivatives
\newcommand{\derivx}[1]{\frac{\mathrm{d}}{\mathrm{d}x} (#1)}
\newcommand{\deriv}[2]{\frac{\mathrm{d} #1}{\mathrm{d} #2}}
% For partial derivatives
\newcommand{\pd}[2]{\frac{\partial #1}{\partial #2}}
% Integral dx
\DeclareMathOperator{\dx}{\mathrm{d}x}
\DeclareMathOperator{\dd}{\mathrm{d}}
% Probability commands: Expectation, Variance, Covariance, Bias
\DeclareMathOperator{\E}{\mathbb{E}}
\DeclareMathOperator{\var}{\mathrm{var}}
\DeclareMathOperator{\Cov}{\mathrm{Cov}}
\DeclareMathOperator{\Bias}{\mathrm{Bias}}
\newcommand{\abs}[1]{\vert #1 \vert}
\renewcommand{\L}{\mathcal{L}}
\DeclareMathOperator{\N}{\mathcal{N}}
\DeclareMathOperator{\plim}{\overset{p}{\to}}
\DeclareMathOperator{\dlim}{\overset{d}{\to}}
\DeclareMathOperator{\vech}{\text{vech}}
\DeclareMathOperator{\vecc}{\text{vec}}
\DeclareMathOperator{\diag}{\text{diag}}

\renewcommand\bar{\overline}
\newcommand{\paren}[1]{\left(#1\right)}
\newcommand{\bracks}[1]{\left[#1\right]}
\newcommand{\curly}[1]{\left\lbrace #1 \right\rbrace}
\newcommand{\pfrac}[2]{\frac{\partial\, #1}{\partial\, #2}}
\DeclareMathOperator{\y}{\mathbf{y}}
\DeclareMathOperator{\x}{\mathbf{x}}
\newcommand{\sign}[1]{\text{sign}\left(#1\right)}



\begin{document}

Outline for estimating HMC model with Gibbs sampling

\section*{Model Parameters:}
\begin{itemize}
  \item $d$: Number of states
  \item $n$: number of observations
\end{itemize}
 
\section*{Notation:}
\begin{itemize} 
  \item $y_{1:k}$: sample of observables from period 1 to $k$
  \item $a_{i:}$: row $i$ of matrix A
\end{itemize}

\section*{Variables:}
\begin{itemize} 
  \item $Y_k$: observable in period $k$. e.g. Inflation rate
  \item $X_k$: State in period $k$. e.g. \{good, bad\}
  \item $A=\{a_{ij}\}$: Transition matrix. $P(X_{k+1} = j|X_{k} = i)$ 
  \item $\rho = \{\rho_i$\}: $\{P(X_0 = i)\}$. Probabiliy of initial states 
  \item $\pi_{k}(s)$ : $P(X_k = s )$. State probabilties in the forward and backward recursions
\end{itemize}

\section*{Distribution for observables:}
\begin{itemize}
  \item $f(Y_k|X_k=i) = \N(\mu_i, \sigma^2_i)$
\end{itemize}
 

\section*{Priors:}
\begin{itemize}
  \item $a_i$: Dir(1,1,\ldots,1)
  \item $\rho$: Dir(1,1,\ldots,1)
  \item $\mu_i| \sigma_i^2$: $\N(\xi_i,\sigma_i^2/\nu_i)$
  \item $\sigma_i^{2}$: $\Gamma^{-1}(\alpha_i,\beta_i)$
  \item $\beta_i$: $\Gamma^{-1}(g,h)$
\end{itemize}


\section*{Hyper Parameter Values:} 
\begin{itemize}
  \item R = max($Y$) - min($Y$)
  \item M = median(Y)
  \item $\xi$ = [M-0.25*R, M, M+0.25*R] 
  \item $\alpha$ = $1_d$
  \item $g$ = $0.2  \ 1_d$
  \item $h$ = $10/R^2 \ 1_d$
  \item $\nu$ = $0.1 \ 1_d$
\end{itemize}


\section*{Conditional Distributions:}
\begin{align}
  \rho| \ldots &\sim Dir\big( 1, 1, \ldots, 1 \big) \\
  a_{i:}| \ldots &\sim Dir\big( 1+n_{i1}, 1+n_{i2}, \ldots, 1+n_{id} \big) \\
  &n_{ij} = \#\{ 1 < k \le n : X_{k-1} =i, X_k=j \} \nonumber \\
  &\text{ i.e. number of transitions from state i to j} \nonumber \\
  (\mu_i, \sigma^2_i) | \ldots &\sim \text{Normal-Inverse Gamma}(\xi, \nu, \alpha, \beta ) \\
  \sigma_i^2 | \ldots &\sim \Gamma^{-1}\left( \alpha+\frac{1}{2}n,\beta + \frac{1}{2} \sum_{k=1}^n(y_k - \bar{y}_k)^2 + 
  \frac{n_i \nu}{2(n_i + \nu)}(\bar{y}_i - \xi_i)^2 \right)\\
  \mu_i| \sigma_i^2, \ldots  & \sim \N\left( \frac{n_i\bar{y}_k + \nu \xi_i}{n_i + \nu }, 
        \frac{\sigma_i^2}{n_i + \nu} \right) \\
  &n_i = \#\{ 1 \le k \le n: X_k = i\} \nonumber  \\
  \beta | \ldots &\sim  \Gamma^{-1}\left( g+\alpha_i, h + \sigma^2 \right)
\end{align}


\section*{Forward Recursion:}
Instead of sampling the states one at a time conditional on the parameters, 
observables and all other states, we calculate the joint distribution of all the
state variables and then sample from the states. This increases the mixing speed
of the chain. We calculate this joint probability recursively with a forwrad-backward
algorithm. 

The forward recusion generates state probabilities based on data only 
up to period $k$ and is used mostly to calculate the terminal state probabilities. 
The forward recursion is calulated as follows. 
\begin{align*}
  p_{krs} &\propto p(X_{k-1}=r, X_{k}=s | Y_{1:k}, \theta) \\
  &= \pi_{k-1}(r|\theta)a_{rs}P(Y_k|X_k = s, \theta)\\
  \sum_{r} &\sum_{s} p_{krs} = 1 \text{ Normalization }\\
  \pi_{k}(s|\theta) &= \sum_r p_{krs}\\
  p_{1rs} &= \rho_r a_{rs} P(Y|X_k = s, \theta) 
\end{align*}

\section*{Stochastic Backward Recursion:}
The stochastic backward recustion calculates the state probabilties 
for each observation based on the entire sample. 

\begin{align*}
  P'_k &= (p'_{kij}) \\
  p'_{krs} &= p(X_{k-1}=r, X_{k}=s | Y_{1:n}, \theta) \\
  p'_{krs} &= p_{krs}\frac{\pi'_k(s | \theta)}{\pi_k(s|\theta)}\\
  \pi'_{k-1}(r|\theta) &= \sum_s p'_{krs}\\
  \pi'_{N}(r|\theta) &= \pi_{N}(r | \theta)
\end{align*}

\section*{Algorithm for drawing $X_k$:}
This nonstochastic backward alogrithm gives us a sequence of states for a given Gibbs 
sweep instead of a sequence of state probabilities. 

\begin{enumerate}
  \item Draw $X_N$ from $\pi_{N}$
  \item Draw $X_{N-1}$ from the Categorical distribution with probabilities
    propotional to the $X_{N}$ column of $P_N$
  \item Iterate backwards until full sample of $X$'s have been sampled
\end{enumerate}

\newpage
\section*{Gibbs Sweep Steps}
\begin{enumerate}
  \item draw $\mu_i$'s and $\sigma_i$'s from Normal-Inverse Gamma distribution 
  via gibbs sampling. 
  \item Update $\beta_i$'s
  \item update $\rho$
  \item update $A$
  \item Forward update $P$ and $\pi$
  \item Backward update $P'$ and $\pi$'
  \item Reorder the states so that the $\mu_i$'s are in increasing order
  \item Update $X$
\end{enumerate}

\section*{Forecasts}
To generate the ``real time'', h-period ahead foreacst for date $k$ with no signals about the future,
we estimate
the model based on data from from the begining of the sample and ending on date $k$. 
After the Gibbs sampler has been run a sufficient number of times we calculate the 
state probabilites by averaging over the state probabilties calculated in the stochastic
backward recursion during each Gibbs sweep. In particular, this gives us the estiamted 
state probabiity for date $k$, $\pi_k(s)$.  We then calculate the probabilty of being
in each state $h$ periods ahead as $\pi_{k+h}(s) = \pi_{k}(s) A^h $  where 
$\pi_{k}(s)$ is a row vector of state probabilties. The point forecast for expected 
infaltion in period $k+h$ is calculated as 
$\E_k(Y_{k+h}) = \sum_{s \in S} \E_k(Y_{k+h}|X_{k+h}=s) \pi_{k+h}(s) = \sum_{s \in S} \mu_s \pi_{k+h}(s) $



\section*{Adding 1 Period Ahead Signals}
In the model where the forecaster's information set in period $k$ is the history 
of inflation and a 1 period ahead noisy signal we would need to make a few changes. 
We would essentially need to run the model for one extra period and treat the signal 
as another observation, making standard adjustments for calculating the mean and variance 
of each state's inflation for the fact that the last observation is a noisy signal. Tbe
forward recursion calculation would also need to be adjusted for the fact that 
in period $k$ we have a noisy signal so the emision probabilty
 $P(Y_{k+1} | X_{k+1}=s, \theta) $ is calculated 
 differently than the first $k$ emision probabilites.

 For the forecasts we now have some information about the state $k+1$ so the model
 will give us an informed estimate for the state probabilties in period $k+1$,
 $\pi_{k+1}(s)$. To calculate the forecast for period $k+h$ we would do the same as 
 before except $\pi_{k+h}(s) = \pi_{k+1}(s) A^{h-1} $. 


\subsection*{Changes Made}

In this section the state indexes $i$ are supressed for clarity. 
The signal structure is assumed to be 
\begin{align*}
  S_k &= Y_k + \epsilon_k \\
  Y_k &\sim N(\mu, \sigma^2)\\
  \epsilon_k &\sim N(0, \kappa \sigma^2)\\
  S_k &\sim N(\mu, (1+\kappa) \sigma^2)
\end{align*}

There are $n$ direct measurments of $y_k$ and $m$ measurements of the noisy signal
$s_k$

The postior for the mean of the observables is given by
\begin{align*}
  \mu | \sigma^2 ... &\sim \N(\zeta, \omega ) \\
  \tau_y &= \frac{n}{\sigma^2} \\
  \tau_0 &= \frac{\nu}{\sigma^2} \\
  \tau_y &= \frac{m}{(1 + \kappa) \sigma^2}\\
  \zeta & \equiv \frac{\tau_y \bar{y} + \tau_0 \xi + \tau_s \bar{s}}{\tau_y  + \tau_0  + \tau_s}\\
  \zeta & \equiv \frac{n \bar{y} + \nu \xi + \frac{m}{1+\kappa} \bar{s}}{n + \nu + \frac{m}{1+\kappa}} \\
  \omega &= \frac{\sigma^2}{n + \nu + \frac{m}{1+\kappa}}
\end{align*}

The postior for the variance $\sigma^2$ is 

\begin{align*}
  \sigma_i^2 &| \ldots \sim \Gamma^{-1}(\tilde{\alpha}, \tilde{\beta}) \\
  \tilde{\alpha} &= \alpha +\frac{1}{2}n + \frac{1}{2}m \\
  \tilde{\beta} &= 
  \beta + \frac{1}{2} \sum_{k=1}^n(y_k - \bar{y}_k)^2 + \frac{1}{2(1+\kappa)}\sum_{k=n+1}^m(y_k - \bar{y})^2 + 
  \frac{\left(n + \frac{m}{1+\kappa}\right) \nu}{2(n + \frac{m}{1+\kappa} + \nu)}(\bar{y} - \xi)^2 
\end{align*}

In the forward recursion step we need to adjust the emission probabilites for the signals
which is straight forwadly done by useing the modified variance of the signals.
\begin{align*}
  y_k | x_k = s, \theta &\sim  \N( \mu , \sigma^2)\\
  s_k | x_k = s, \theta &\sim  \N( \mu , (1+\kappa)\sigma^2)
\end{align*}

\end{document}

