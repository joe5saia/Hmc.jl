\documentclass[12pt]{article}
\usepackage{fancyhdr}
\usepackage{xfrac}
\usepackage{extramarks}
\usepackage{amsmath}
\usepackage{amsthm}
\usepackage{amsfonts}
\usepackage{graphicx}
\usepackage{color}
\usepackage{pdfpages}
\usepackage{natbib}
\usepackage{caption}
\usepackage{subcaption}
\usepackage{hyperref}
\hypersetup{
    colorlinks=true,
    linkcolor=black,
    filecolor=black,      
    urlcolor=black,
    citecolor=black,
}
\usepackage{booktabs}
\usepackage{geometry}

\geometry{margin=1in}


%
% Basic Document Settings
%

%\topmargin=-1.5in
%\bottommargin=1.5in
%\evensidemargin=1.5in
%\oddsidemargin=1.5in
%\textwidth=6.5in
%\textheight=9.0in
%\headsep=0.25in

\linespread{1.5}

\pagestyle{plain}

% Paragraph indent length
%\setlength\parindent{0pt}

% Custom definitions
% For derivatives
\newcommand{\derivx}[1]{\frac{\mathrm{d}}{\mathrm{d}x} (#1)}
\newcommand{\deriv}[2]{\frac{\mathrm{d} #1}{\mathrm{d} #2}}
% For partial derivatives
\newcommand{\pd}[2]{\frac{\partial #1}{\partial #2}}
% Integral dx
\DeclareMathOperator{\dx}{\mathrm{d}x}
\DeclareMathOperator{\dd}{\mathrm{d}}
% Probability commands: Expectation, Variance, Covariance, Bias
\DeclareMathOperator{\E}{\mathbb{E}}
\DeclareMathOperator{\var}{\mathrm{var}}
\DeclareMathOperator{\Cov}{\mathrm{Cov}}
\DeclareMathOperator{\Bias}{\mathrm{Bias}}
\newcommand{\abs}[1]{\vert #1 \vert}
\renewcommand{\L}{\mathcal{L}}
\DeclareMathOperator{\N}{\mathcal{N}}
\DeclareMathOperator{\R}{\mathcal{R}}
\DeclareMathOperator{\plim}{\overset{p}{\to}}
\DeclareMathOperator{\dlim}{\overset{d}{\to}}
\DeclareMathOperator{\vech}{\text{vech}}
\DeclareMathOperator{\vecc}{\text{vec}}
\DeclareMathOperator{\diag}{\text{diag}}

\renewcommand\bar{\overline}
\newcommand{\paren}[1]{\left(#1\right)}
\newcommand{\bracks}[1]{\left[#1\right]}
\newcommand{\curly}[1]{\left\lbrace #1 \right\rbrace}
\newcommand{\pfrac}[2]{\frac{\partial\, #1}{\partial\, #2}}
\DeclareMathOperator{\y}{\mathbf{y}}
\DeclareMathOperator{\x}{\mathbf{x}}
\newcommand{\sign}[1]{\text{sign}\left(#1\right)}



\begin{document}

\section{Notation}

\begin{enumerate}
  \item $i \in \{1,2,3\} $ denotes the state of the Markov process
  \item $Y_t$ is an observation of inflation at time $t$
  \item $X_t \in \{1,2,3\}$ is the state of the Markov process at time $t$
  \item $\Theta$ is a vector of all parameters; state means, state variances, transistion probabilties, 
  and a $T\times3$ matrix of state probabilties for each period ($\mu, \sigma^2, A, \pi$)
  \item $\Phi(\cdot)$ is the CDF of the standard Normal
  \item $F_t$ is the belief of the agent at time $t$ for the distribution of the parmaters 
\end{enumerate}

We have the following by definition of the Markov process
\[
  Y|i \sim \N(\mu_i, \sigma^2_i)
\]

\begin{align*}
  Pr_t(Y_t \leq y) &= \int_{\theta \in \Theta}Pr(Y_t \leq y|\theta)\dd F_t(\theta) \\
  Pr_t(Y_t \leq y) &= \int_{\theta \in \Theta} \Phi\left( \frac{y-\mu}{\sigma} \right) \dd F_t(\theta) \\
  Pr_t(Y_t \leq y) &= \E_{\theta \sim F_t} \left[ \Phi\left( \frac{y-\mu}{\sigma} \right)  \right]
  \intertext{When $t \leq T$, we have }
  \E_{\theta \sim F_t} \left[ \Phi\left( \frac{y-\mu}{\sigma} \right)  \right] &=
      \E_{\theta \sim F_t} \sum_{i=1}^3 \Phi\left( \frac{y-\mu_i}{\sigma_i} \right) \pi_t(i)
\end{align*}

Now we want a function $f : \R \times \Delta(\Theta) \to \R $ such that for a given $F \in \Delta(\Theta)$,
$f$ maps a standard normal to $Z$. Let $Z \sim \N(0,1)$, then we want a function $f$ such that 
\[
 Y_t = f(Z; F_t)  
\]

Note that 
\begin{align*}
  Pr_t(Y_t \leq y) &= Pr(f(Z; F_t) \leq y) \\
  Pr_t(Y_t \leq y) &= Pr(Z \leq f^{-1}(y; F_t)) \\
  Pr_t(Y_t \leq y) &= \Phi\left(f^{-1}(y;F_t)\right)
\end{align*}

Combining results we have 

\begin{align*}
  \Phi\left(f^{-1}(y;F_t)\right) &= \E_{\theta \sim F_t} \left[ \Phi\left( \frac{y-\mu_i}{\sigma_i} \right)  \right]
  \intertext{Implying}
  f^{-1}(y;F_t)&= \Phi^{-1}\left(  \E_{\theta \sim F_t} \left[ \Phi\left( \frac{y-\mu}{\sigma} \right)  \right] \right) \\
  f^{-1}(y;F_t)&= \Phi^{-1}\left( \E_{\theta \sim F_t} \sum_{i=1}^3 \Phi\left( \frac{y-\mu_i}{\sigma_i} \right) \pi_t(i) \right) \\
\end{align*}


\begin{equation}
  \label{eqn:1}
  \E_{\theta \sim F_t} \sum_{i=1}^3 \Phi\left( \frac{y-\mu_i}{\sigma_i} \right) \pi_t(i)
\end{equation}
can be estimated via gibbs sampling. For each gibb's sweep calculate the term
\[ 
  \sum_{i=1}^3 \Phi\left( \frac{y-\mu_i}{\sigma_i} \right) \pi_t(i)
\]
and average over the sample of these draws. The function $f^{-1}$ can then be estimated 
by calculating equation \ref{eqn:1} over a grid of $y$.

\newpage
\section{Biases}

We want to decompose the bias in the forecast for inflation. Following the above notation with $y_t$ denoting inflation 
at time $t$ and $\pi_{it}$ denoting the probability of being in state $i$ at time $t$. The agent has a belief 
about the probability of being in each state for every date in the data (This is true even within a single Gibb's sweep).
The probability vector for a future state is the probability of transistioning to each future state from the 
current state(rows of $A^h$), weighted by the probability of being in each current state.
The inflation forecast is defined as:
\begin{align}
  \pi_{t+h} & = \pi_t A^h  & \text{State forecast} \\
  \E_t[y_{t+h}| \theta \sim F_t] &= \E_t[\pi_{t+h}' \mu| \theta \sim F_t] 
  \intertext{Replacing the time subscript with the conditional distribution for notational reasons}
  \E_{F_t}[y_{t+h}] &= \E_{F_t}[\pi_{t+h}' \mu] 
\end{align}
$y_{t+h}$ is a function of $\pi_{t+h}$ and $\mu_t$. We can take a first order talyor expansion
and then calculate the variance of it to give (similiar to delta method techniques):
\begin{align}
  \E_{F_t}[y_{t+h}] &\approx \E_{F_t}[\pi_{t+h}'] \E_{F_t}[\mu] + \nabla_{\theta}(y_{t+h})' \Cov_{F_t}(\theta) \nabla_{\theta}(y_{t+h})
  \label{bias} \\
  \theta &= [\pi_{t+h}', \mu']' \\
  \nabla_{\theta}(y_{t+h})' &= [\mu_1, \mu_2, \mu_3, \pi_{1, t+h}, \pi_{2, t+h}, \pi_{3, t+h}]
\end{align}
Where the derivatives are evalulated at the true values of of $\theta$

The second term is the bias comming from model uncertainty. This term can be estimated 
by replacting the terms in $\nabla_{\theta}(y_{t+h})'$ with the expected values across the 
Gibbs draws. The covariance term can also be calculated by calculating the covariance of the
terms across Gibbs draws. 

\subsection{Notes}
In the above derivation, we collapsed the current state probabilties and transition matrices into 
a single vector of future state probabilties. We can unwind this and expand the $\theta$ vector,
but this complicates the $\nabla_{\theta}(y_{t+h})'$. This is doable just more complicated. 

\subsection{Questions}
\begin{enumerate}
  \item Does this look correct?
  \item What do you mean by ``correlate with the convexity of $f^{-1}$?'' Specfically, do you have a convexity measure in mind?
\end{enumerate}

\end{document}

